\documentclass[12pt]{article}
\usepackage[utf8]{inputenc}
\usepackage{array}
\usepackage{xcolor}
\usepackage{graphicx}
\usepackage{mathtools}
\usepackage{amsmath}
\usepackage{multicol}
\usepackage{eqnarray}
\usepackage{wrapfig}


\usepackage{natbib}
\usepackage{hyperref}
\hypersetup{
	colorlinks=true,
	linkcolor=black,
	filecolor=mangeta,      
	urlcolor=blue,
	pdftitle={Overleaf Example},
	pdfpagemode=FullScreen,
}

\usepackage[margin=0.6in]{geometry}

\title{52nd—24th INTERNATIONAL-RUDOLF ORTVAY \\ PROBLEM SOLVING CONTEST IN PHYSICS \\ Problem 14}
\author{Nguyen Thanh Long}
\date{\today}

\begin{document}
	
\maketitle
	
\noindent Using the Ampere's law, we know that the magnetic field inside the soleloid (by the soleloid) is 
\begin{equation} \label{eq1}
	B_0 = \frac{\mu_0 N I}{l} .
\end{equation}	
\noindent We can denote easily that in the homogeneous magnetic field, a tiny small iron sphere with relative permeability $\mu_r$ will have the magnetic dipole \cite{1} :
\begin{equation} \label{eq2}
	m = V \frac{3 \left( \mu_r - 1 \right) B_0 }{\mu_0 \left( \mu_r + 2 \right) } .
\end{equation}	
\noindent The magnetic flux that the dipole effect on a circle wire distance $z$ from the dipole is:
$$ \frac{ d \Phi (z) }{d N} = \int_0^r \frac{\mu_0}{4 \pi}  \frac{ 3 z^2 - \left( z^2 + r^2 \right) }{ \left( z^2 + r^2 \right)^{\frac{5}{2}} } \cdot 2 \pi r dr = \frac{ \mu_0 m r^2 }{2 \left( z^2 + r^2 \right)^{\frac{3}{2}} } .$$
\noindent Where $dN = \frac{N}{l} dz$, so the total magnetic flux of the soleloid (caused by dipole) will be:
\begin{equation} \label{eq3}
	\Phi = \int_{-\infty}^{\infty} \frac{N}{l} \frac{ \mu_0 m r^2 }{2 \left( z^2 + r^2 \right)^{\frac{3}{2}} } dz = \frac{ N \mu_0 m}{l} .
\end{equation}	

\noindent From (\ref{eq1}), (\ref{eq2}) and (\ref{eq3}), we have:
$$ \Phi = \frac{ 3 \mu_0 \left( \mu_r -1 \right) V N^2}{\left( \mu_r + 2 \right) l^2 } I .$$

\noindent Therefore, the self-inductance change of the solenoid is;
$$ \Delta L = \frac{ 3 \mu_0 \left( \mu_r -1 \right) V N^2}{\left( \mu_r + 2 \right) l^2 } .$$
\noindent With $\mu_r \gg 1$, the result will be: 
$$ \Delta L = \frac{ 3 \mu_0 V N^2}{l^2} .$$

\bibliographystyle{plain}
\begin{thebibliography}{1}
	\bibitem{1} J. D. Jackson, \textit{Classical Electrodynamics}, 3rd ed. (John Wiley \& Sons, New York, NY, 1998).
\end{thebibliography}	
	
\end{document}